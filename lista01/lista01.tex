\documentclass[a4paper]{article}

\usepackage{fullpage} % Package to use full page
\usepackage{parskip} % Package to tweak paragraph skipping
\usepackage{tikz} % Package for drawing
\usepackage{amsmath}
\usepackage{hyperref}

\title{Lista de exercícios - Algoritmos}
\author{Waldeyr Mendes Cordeiro da Silva\\IFG, Formosa, 2019-1}

\begin{document}

\maketitle

\section{Lista 01}

\begin{enumerate}
	\item Fazer um programa que leia um número inteiro e verifique se é divisível simultaneamente por 3 e 5.
	\item fazer um programa que leia 4 números inteiros e realize a soma dos 3 maiores.
	\item Fazer um programa qque leia o número de uma sala de aula, o número de salas e calcule a quantidade de alunos na escola.
	\item Desenvolver um algoritmo que leia coeficientes $a$, $b$ e $c$ de uma equação do segundo grau e calculee imprima suas raízes, quando possível.
	\item Fazer um programa para calcular a conta de água, sabendo que o custo da água varia se o consumidor é residencial, comercial ou industrial de acorso com as seguintes restrições:
	\begin{itemize}
		\item Residencial: R\$ 5,00 de taxa mais R\$ 0,05 por m$^3$ gastos;
		\item Comercial: R\$ 500,00 para os primeiros 80 m$^3$ gastos mais R\$ 0,25 por m$^3$ gastos;
		\item Industrial: R\$ 800,00 para os primeiros 100 m$^3$ gastos mais R\$ 0,04 por m$^3$ gastos;
	\end{itemize}
	\item Fazer um programa que encontre o maior dentre 3 números quaisquer, verificando se todos são distintos.
	\item Fazer um programa que calcule seu peso ideal de uma pessoa, tendo como dados de entrada a altura e o sexo, utilizando as seguintes fórmulas:
	\begin{itemize}
		\item Para Homens: (72.7*h) – 58;
		\item Para Mulheres: (62.1*h) – 44.7
	\end{itemize}
\end{enumerate}

\end{document}